Purpose, scope and governance
Digital sustainability is the practice of using digital technology and solutions in a way that ensures long term environmental, economic and social sustainability. It is sometimes referred to as ‘green IT’ (information technology) or ‘sustainable ICT’ (information and communication technology). This definition recognises the ‘green IT’ or ‘IT for green’ dilemma – namely that technology can help address the climate crisis whilst also being a contributor to it.

The purpose of this strategy is to set out Defra group vision to minimise the environmental, economic and social impact of its ICT operations. It simultaneously contributes to overall enterprise sustainability goals and unlocks associated benefits. The strategy encompasses all aspects of ICT, including (but not limited to):

data centres
cloud services
digital workplace devices
software
data
supply chain and vendor relationships
The document outlines how this vision will be achieved over the next 5 years. In doing so, it serves several purposes as it:

stops us over-committing to poorly managed and uncoordinated activity and gives others the chance to challenge our priorities if needed
enables people working in Defra group to align their work and decisions with the strategy’s actions and objectives
helps suppliers and other external stakeholders to understand our approach, so that they can tailor their work with us to be as effective as possible
enables us to be held accountable for making progress on our stated goals and milestones for this important issue
brings our policy priorities and corporate commitments on digital sustainability into one place, so that they can be more easily understood and delivered
The intended audience of this strategy is primarily staff working on Defra digital services, including suppliers and commercial, who will be responsible for many of the actions included. Colleagues in wider Defra group and government may find it relevant for information purposes. The strategy does not set out actions for individual users, but rather the organisational approach that Defra group will take.

This strategy covers all of Defra group’s corporately provided technology and digital services. This includes:

core Defra
Environment Agency (EA)
Natural England (NE)
Marine Management Organisation (MMO)
Rural Payments Agency (RPA)
Animal and Plant Health Agency (APHA)
There is a separate strategy for cross government digital sustainability, reflecting Defra’s leadership in this area.

The strategy:

is intended to be the comprehensive and authoritative plan for our digital sustainability approach from 2025 to 2030
is not intended to replace or take precedence over any other existing strategies, but does provide further detail on how we intend to deliver on commitments made in many of them
This strategy does not provide additional funding beyond anticipated baseline budgets. The activities outlined are achievable within expected Spending Review settlements. Many sustainability challenges link with key business and commercial challenges. Sustainability can be a key part of the solution enabling cost savings and business resilience. The objectives outlined in this document:

introduce efficiency
eliminate unproductive resources
apply circularity
Therefore, full exploitation of opportunities will have the parallel advantage of delivering cost optimisation.  

Defra’s Digital Sustainability team will monitor delivery of this strategy. They report into Defra’s Digital, Data, Technology and Security Executive Board and upwards into Defra group Corporate Services Board. This governance route has also been used for approval of this strategy. We will adapt our governance of this accordingly in line with evolving best practices and regulations.

Vision and objectives
Our vision is that by 2030, Defra group’s technology and digital services are sustainable, secure and resilient by design. Sustainability risks and opportunities are understood, acted on and monitored. The key performance indicators outlined for each of the objectives below will be used to monitor our progress to this vision.

To deliver our vision, we have set out 6 strategic objectives. These are:

Reduce and mitigate carbon emissions towards net zero targets.
Reduce the wider planetary impacts of digital services (such as water use and e-waste).
Reduce natural resource use and improve our circular economy approach.
Reduce social risk and deliver social value.
Increase transparency and accountability, internally and in the supply chain.
Improve our technology’s resilience to climate and environment risks.
These are underpinned by a cross-cutting objective to:

7. Embed digital sustainability as business as usual.

Further detail on each of the objectives is provided in the sections below, including (for each objective):

what we need to understand and demonstrate
activities that will help us to operationalise and deliver the objective
key performance indicators (KPIs) we will use to keep us accountable
our intended outcome by 2030
Policy priorities and corporate commitments   
Policy priorities
In July 2024, Defra’s Secretary of State set out following 5 main priorities for Defra to support the Government’s Plan for Change missions:

To clean up Britain’s rivers, lakes and seas.
To support farmers to boost Britain’s food security.
To improve our nation’s resilience to the dangers of flooding.
To ensure nature’s recovery.
To maximise the value of resources.
This strategy responds to these priorities, with a particular focus on the last two. Digital sustainability can be a key driver that helps to support a zero-waste economy, particularly if we adopt a more circular approach.

Digital sustainability is also important for nature’s recovery. For example, digital operations can have significant impacts on biodiversity through poorly managed water or land use. Conversely, many technology suppliers support nature recovery schemes as part of their technology contracts through their social value commitments.

Work to lever technology to achieve positive outcomes across all priorities is covered elsewhere in Defra group’s strategy and Defra Digital, Data, Technology and Security (DDTS) teams elsewhere.

Corporate commitments
Greening Government Commitments (GGCs)
Like all departments, Defra is required to deliver the Greening Government Commitments. The next set of commitments is currently being reviewed to ensure they remain aligned with government policy.

Defra group sustainability strategy
The Defra group sustainability strategy commits Defra group to reaching net zero by 2050. Defra group’s IT emissions make up an estimated 13% of its overall emissions in 2024. Our arm’s length bodies also have individual sustainability strategies with relevant commitments, for example the Environment Agency’s own net zero strategy.

Defra group’s Digital and Data and Transformation Strategy
Defra group’s Digital and Data and Transformation Strategy commits to the following under Mission 4 (Efficient, secure, and sustainable technology and services):

Defra group will have full visibility of, and insight into, the environmental impact of our information and communication technology
Defra will be a recognised role model in sustainability of technology and exemplar to other government departments and to industry
digital will enable Defra’s net zero target as well as Environment Agency’s net zero target
Defra group supply chains will be free of slavery and exploitation
we will have world leading sustainability process
Defra group will use sustainable technology to deliver the Environment Improvement Plan and 25-year Environment Plan on behalf of UK government
Defra’s Chief Data Office has developed a 2030 Data and Information Strategic Roadmap that stems from the Digital Strategy and sets out minimum expectations for suppliers and partners.

DDTS operating model
DDTS new Directional Design sets out how DDTS will operate by 2030. It commits to a consistent application of sustainable IT principles in design and delivery of services.

Policy Procurement Notes (PPN) and National Procurement Strategy
PPN 006: requires a commitment to net zero and a carbon reduction plan for procurements over £5 million in value
PPN 01/24: an optional contractual schedule for decarbonisation objectives to be set out and delivered
PPN 002: a mandatory requirement from 1st October 2025 for procurements to apply the social value model
PPN 009: guidance for how modern slavery risks are identified and managed in procurement
National Procurement Policy Statement: sets out strategic priorities for public procurement
This strategy will define steps that help us deliver on these commitments, while striving for even greater impact.

The context for the strategy
Progress in Defra group since 2020  
In 2020, Defra group published its first ever Sustainable Information Technology (IT) Strategy. This provides a high level view of its approach to meeting the original  Greening Government: ICT and Digital Services Strategy 2020 to 2025.

Since then, Defra has largely delivered on the objectives of this original strategy. For example, we now report annually on our sustainable IT performance and have embedded sustainability into our project assurance and technology procurement processes. In particular, our recent procurements for major infrastructure contracts have been recognised as leading the way for sustainable technology procurements across government.

But we still have a lot to do. For example, in 2023 to 2024 Defra group was responsible for 198 tonnes of e-waste; only 40% of which was re-used. During the same period Defra group IT was also responsible for nearly 10,000 tonnes of carbon emissions. Both our e-waste figures and IT emissions have shown increases in recent years, making the challenge we face more pressing.

Looking forward in Defra group
The period from 2025 to 2030 is an exciting one for Defra. Following a comprehensive procurement phase, we will see the launch of 4 new major contracts over the next few years:

connectivity (Wi-Fi, WAN, LAN and Gateways)
end user services (laptops, devices, service desk)
hosting and storage (digital private cloud, platform services, infrastructure services and application services)
application management and support (provision of IT infrastructure services)
These new contracts are the amongst the first in Defra to take a new approach to sustainability, with robust sustainability requirements embedded across all of them. They will deliver further opportunities to embed digital sustainability across our operations.  We will improve and standardise our supplier engagement, engaging the DDTS service manager and owners to ensure these opportunities are realised, and commitments are met.

After a review to ensure alignment with government priorities, we will also see the launch of the new Greening Government Commitments framework (to 2030). These are expected to establish an updated series of stretching targets on emissions, water and waste that UK government departments and their agencies will be challenged to meet. This will include an ambitious set of targets for digital and technology (ICT) services. Sustainable procurement will also be a key focus of these commitments.   

In the next 5 years we can expect to see AI (artificial intelligence) to play an increasing role across Defra’s estate. This presents an enormous opportunity for us, yet puts pressure on resources such as electricity, water and precious minerals as the industry grows to accommodate demand. These risks are difficult to quantify due to the lack of transparency in the AI industry. To maximise opportunities within Defra’s estate, we need to proactively manage the risks of AI as they relate to sustainability.

Important digital sustainability challenges
Carbon emissions
[footnote 1]

The global ICT carbon emissions footprint already exceeds that of aviation and is growing faster than the rate of global emissions [footnote 2]. The primary sources of ICT sector emissions are:

end-user devices (mainly their extraction, manufacture and distribution)
data centres (both on-premise and cloud)
connectivity networks (mobile and fixed)
Due to the complex nature of the ICT supply chain, measuring emissions across the full digital lifecycle can be challenging.

In Defra group, our 2023 to 2024 IT emissions are 9,900 tonnes from centrally managed IT and 87,540 tonnes from Scope 3 emissions from our suppliers. This figure makes up 13% of Defra group’s total footprint and therefore is a significant area of opportunity for reduction.

Wider planetary impacts – for example water use and biodiversity loss
[footnote 3]

Digital technology is a highly water intensive industry, with water required across multiple points in the lifecycle of a digital service. At the start of the lifecycle, the mining of raw materials required for digital devices requires large quantities of water. Water is also required in the manufacturing of the devices. Both the mining and manufacture can create toxic pollutants that contaminate water and land sources. When devices move into use they consume electricity, with water being required for the electricity generation process. The rising use of artificial intelligence (AI) and cloud computing services will put further pressure on water resources – especially in water stressed regions.

In Defra, we used 25,484,993 kwh of electricity to power our hardware and data centres, which consumed millions of litres of water to produce. The data centres used to host our services also require water for cooling. Although we don’t report water usage for hosting in Defra, we know this is an area of concern and are working with our supply chain to improve transparency.

The extraction, manufacture and disposal of digital devices can also cause to habitat loss, nature destruction, pollution and biodiversity decline.

Circularity and electronic-waste generation
[footnote 4]

Rapid development of technology devices and services and the short lifecycle of these devices means that large quantities of electronic waste (e-waste) are produced. E-waste can be challenging to refurbish or recycle, with 80% globally going to landfill where it is environmentally harmful due to its lead, mercury and cadmium contents. E-waste is one of the fastest growing solid waste streams globally – in 2022 an estimated 62 million tonnes of e-waste were produced. In Defra 2023 to 2024 we were responsible for 198 tonnes of e-waste; only 40% of which was re-used.

Resource extraction
[footnote 5] [footnote 6]

The production and manufacture of digital devices and hardware involves the extraction of rare earth elements and other valuable minerals. These elements and minerals are finite within the earth. Mining them is both unsustainable due to the finite supply, and causes environmental damage such as land contamination, ecosystem disruption and biodiversity loss. Our reliance on these minerals is an economic business resilience risk, as their finite nature means future shortages are expected. Developing a circular approach to our hardware will minimise the quantity of virgin resources needed.

Rebound effects
ICT can reduce emissions in other sectors, yet it can also facilitate rising emissions in other sectors as efficiency improvements are eroded by growth in demand [footnote 7]. Although this has not been mapped in Government, in Defra this may occur as a service becomes more accessible through digitisation, leading to a higher uptake of users. We may also see AI and increased access to cloud computing facilitate a faster rolling out of services, increasing the size of our digital portfolio.

Social risks and impacts
[footnote 8] [footnote 9] [footnote 10] [footnote 11]

Several different social risks exist within the digital and technology supply chain. The supply chain is high risk for modern slavery (such as forced and child labour) and wider violations of labour rights. This is particularly the case in mining regions and manufacturing hubs. Many of the essential minerals required (such as ‘3TG’ minerals: tantalum, tin, tungsten and gold) are extracted in conflict zones, with workers facing poor and exploitive labour conditions. They are mined and processed in areas of known environmental destruction, human rights abuses and civilian suffering. Outsourced services such as call-centres, waste and recycling also have EDI (equality, diversity and inclusion) risks across the sector.

Those who benefit from the digitalisation are typically not the same as the people who bear the social cost or the negative environmental impacts. There is unequal access to technology in a ‘digital divide’ meaning not all equally benefit from its advantages.

Due to the complexity of technological supply chains, it’s difficult for organisations like Defra to have full transparency of the social risks in the technology we procure.

Supply chain transparency
A risk to managing the above environmental, social and economic impacts is the lack of transparency in the global technology supply chain [footnote 12]. Digital technologies, including our end user devices at Defra and the components that make up our data centres, are manufactured from hundreds of components sourced from different suppliers across the globe. Once manufactured there are then more layers in the supply chain before it reaches an employee’s desk, or a data centre. Each component may have different environmental and social impacts, but these are obscured due to the lack of supply chain visibility beyond the first supplier. This risk is increased for infrastructure outsourced to suppliers, such as in Defra’s cloud hosting, where shared service models make it difficult to allocate emissions and identify risks. Working with Defra group’s technology supply chain to improve transparency around key sustainability risks is key to mitigating this.

Resilience
Climate resilience is a critical risk in digital services. Our services are reliant on data centres, which require 24/7 uninterruptible power supply to be operational. As the climate changes and heat waves and floods become more likely, services are at increasing risk of disruption. This was exemplified in 2022 when NHS IT servers ceased working during a heatwave, leading to widescale disruption across 2 hospitals [footnote 13]. In Defra, we require suppliers to have business continuity plans in place so that appropriate mitigations are in place.

The actions we will take
We will work with subject matter experts to assess the impact and complexity of the actions below to prioritise and sequence them over the next 5 years. Some of the objectives below may already be underway but have been included here so there is a comprehensive overview of all actions.

Many of the actions below will involve our supply chain. We recognise that some of the actions may be challenging for SMEs (small and medium sized enterprises) and VCSEs (voluntary community and social enterprise) to meet. This is as SMEs and VSCEs don’t have the resources available that larger organisations do. As innovation and good practice are often driven by SMEs and VSCE’s we want to encourage their participation. We have endeavoured to ensure the actions below are proportionate and give SMEs and VSCEs the flexibility to form part of our supplier base. We will continue to work with our supply chains to inform the selection of proportionate and non-discriminatory questions and criteria for social value and sustainability in our procurements, with continued regard to our objectives in section 12 of the Procurement Act 2023.

SO1: reduce carbon emissions towards net zero targets
What we need to demonstrate
How Defra’s digital and technology services and operations may contribute towards carbon emissions (scopes 1, 2 and 3), and influence our capacity to reach net zero targets.  We also need to know what can be done to reduce and mitigate the impacts where possible, helping to deliver Defra’s net zero targets.

Activities that will help us to deliver this objective by 2030
Emission transparency and baselining:

collect data on ICT emissions, to enable us to track and analyse carbon emissions from our services, as required by the Greening Government Commitments (GGC) annual reporting
work with suppliers to improve transparency in, report and track scope 3 emission data as we know this is where a significant proportion of our carbon emissions are from
use this data to baseline DDTS’s full carbon footprint, including value chain emissions
develop a methodology for and provide service-level emissions reporting for Defra group’s top digital services
develop carbon personas and a personal carbon calculator, so each staff member in Defra understands how their activities contribute to the overall footprint
Advice and guidance to projects and services:

develop and implement a ‘Greener Service Principles (GSPs)’ framework to help guide design, delivery and operation of services to have a lower carbon footprint - for example:
lightweight digital pages: ensure web pages and digital content have minimal file sizes by reducing unnecessary media and animation and optimising images and files
minimal dependencies: services to limit external dependencies such as external libraries and plug ins to reduce page bloat and data transfer
notification minimisation: ensure services send minimal notifications and notifications are kept lightweight, set up notifications so that users can turn them off or choose how to receive them
work with delivery group leads to understand future project demand and influence direction
use the sustainability assessment process to signpost teams to guidance on relevant actions they can take to reduce their carbon footprint, such as providing the GSPs and other relevant documentation like the AI position statement and principles
support the project teams with measurement of their services, with a longer-term aim of creating carbon KPIs for services
Responsible procurement and supplier management:

where proportionate and relevant to the procurement:

procure from suppliers that have a published net zero strategy that meets established best practice (for example, Science Based Targets Initiative - SBTI) and can support us in delivering our own net zero strategy
include requirements for technology suppliers to include transparent Scope 1 to 3 emissions reporting
manage suppliers to monitor sustainability performance and commitments
Digital workplace and hardware:

move to provision of refurbished end user devices by default
government buying standards and Ecolabels: where new products must be procured, ensure all purchases comply with the relevant government buying standard. If there is no applicable buying standard, device should be procured complying with a relevant ecolabel (such as TCO, EPEAT and Energy Star). Where feasible, accompanying PCFs (product carbon footprints) or equivalents should be provided by the manufacturer.
repair and refurbishment: ensure we have capability to repair and refurbish end user devices to prolong their life
analytics-driven performance-based device refreshes: ensuring we are extending device lifetimes to their full potential, rather than implementing standard lifecycle refreshes
estate rationalisation: ensuring our hardware estate is a small as possible. Limit the number of Defra devices to a minimum requirement by optimising desk set ups and leveraging technologies such as virtual workstations, remote desktops and video conferencing software
power management: ensure all hardware contains power management configurations that are fully utilised
remove unused equipment: ensure hardware and software that are unused are removed from the estate
create a roadmap group to review the future needs of end user needs and sustainability opportunities in this space
Data centre optimisation:

Through procurement and work with suppliers we will ensure:

uninterruptible power supply (UPS) right sizing: ensuring UPS systems are appropriately sized to reduce energy waste while maintaining resilience
PUE (Power Usage Effectiveness): PUE is continually improved through improved cooling, airflow, and infrastructure efficiency
ERF (Energy Reuse Factor): encourage suppliers to measure ERF (amount of heat reused from waste heat relative to total energy) to encourage the adoption of energy reuse practices
CER (Cooling Efficiency Ratio): encourage suppliers to measure CER (assesses efficiency of cooling systems) to optimise their cooling systems
high server utilisation and compute efficiency – ensuring suppliers optimise resource allocation, consolidate workloads, and implement dynamic balancing to minimise idle infrastructure
virtualisation: ensuring we reduce the number of physical servers by using virtualisation
renewable energy: encourage data centres to be utilising renewable energy sources to supply power, ideally through on-site generation. Where on-site generation is not feasible, suppliers should be investing in PPAs (Power Purchase Agreements) to finance renewable projects
industry best practice: ensure suppliers are aligning with industry best practice, such as by joining the EU Code of Conduct for Data Centres and integrating with ISO standards on energy and environmental management
decommission legacy servers: ensure servers that are no longer required are decommissioned
Cloud optimisation: 

Ensure we are utilising the full benefits that public cloud has to offer by driving GreenOps including:

right-sizing and resource optimisation: ensure our services are efficiently provisioned, with auto-scaling in place to dynamically adjust capacity based on demand
review aged instances and decommission unused resources
apply green software engineering principles in building and maintaining applications
maximising benefits of cloud-native applications and services – leverage cloud-native architectures by designing applications that take full advantage of the services that public cloud offers
transitioning from ‘always on’ to ‘always available’ – moving towards architectures where compute resources automatically scale to zero when not in use, ensuring idle workloads do not consume unnecessary energy. Actively turning off compute instances when they are no longer needed
tiered storage implementation – assigning different types of data to the most energy-efficient storage media, such as using cold storage for archival data and high-performance storage only when required to reduce energy-intensive operations
where suitable, leveraging tools available from suppliers to review and improve services
Addressing our legacy application portfolio:

work with the legacy application strategy and programme to ensure sustainability is a consideration when modernising the legacy technology estate
create sustainability impact assessments for legacy remediation to understand the sustainability risks, benefits and opportunities to reduce carbon costs
use tools such as CAST to identify areas of improvement in legacy applications
Data & Information:

work with the data teams and others to ensure roles are clearly understood and that we remove obsolete data, limit data duplication and ensure data retention policies
ensure existing data and information policies are adopted, and continuously improved
ensure Defra’s data and information services are reused and enhanced where applicable
ensure data has a data owner, aligned with Defra’s data ownership policy, to ensure accountability and leadership in managing Defra’s data across the lifecycle
manage information appropriately throughout its lifecycle, ensuring deletion when no longer required. Comply with information policies and procedures for your organisation and work with the Information Management team (or equivalent) when required
Key Performance Indicators (KPIs):
16% reduction in Defra group’s IT carbon footprint by 2030 from 9,900 tonnes of centrally managed IT and 87,540 tonnes from our supply chain.
100% of suppliers of digital services with a contract value of £1 million per annum or above have externally verified carbon footprints and a plan to achieve net zero by 2050 (or sooner unless Defra determines that it would not be relevant or proportionate to do so for a specific contract in line with its obligations under the Procurement Act 2023).
PUE is no higher than 1.3 in all data centres we operate in.
Outcome for 2030:
We understand our contribution to organisational carbon emissions. We are taking proactive steps (both within DDTS, and though our suppliers) to reduce emissions to work towards meeting the Government’s net zero target of 2050. This would equate to a 16% reduction over the next 5 years.

SO2: reduce the wider planetary impacts of digital services
What we need to demonstrate
We need to understand how Defra’s digital and technology services interact with and contribute to the environmental impacts beyond carbon and put mitigations in place accordingly. This includes addressing water consumption, resource use, land use change, biodiversity and pollution.   

Activities that will help us to deliver this objective by 2030
Minimise the wider planetary impacts of digital products and services:

develop our sustainability assessment process to include wider planetary impacts of digital products and services
support project teams in assessing and mitigating digital services impact on global land use change, nature recovery, and biodiversity
support project teams to better understand and mitigate their services contribution to adverse or excessive use of water and resources
Improve the wider planetary impacts of the supply chain:

work with suppliers to understand the impact of water consumption in our data centres through measurement and reporting of metrics such as total water consumption and WUE (Water Usage Effectiveness). Implement water metrics reporting as part of relevant procurements
work with suppliers to mitigate water consumption in data centres, such as through introduction of closed loop cooling
work with suppliers to better understand and minimise resource depletion, e-waste and introduction of pollutants into the environment in the delivery of their services
encourage suppliers to adopt TNFD (taskforce on nature-related financial disclosures) to better incorporate nature into their decision-making
encourage suppliers to commit to nature recovery and biodiversity improvement through social value commitments (such as habitat creation, biodiversity enhancement).
conduct LCAs (life cycle assessments) and where feasible request product passports for hardware to enable informed, sustainable purchasing decisions
Key Performance Indicators
Track the quantity of nature recovery activities delivered as part of supplier social value commitments and action plans.
Where relevant and proportionate implement reporting requirements on resource (waste) and water use as part of contractual obligations with DDTS strategic suppliers [footnote 14], with an aim of 100% of contracts over £1 million reporting on resource use, waste and water by 2030.
By 2030 where relevant and proportionate ensure LCAs are included as part of reporting requirements, for 100% of contracts over £1 million where hardware is procured, appropriate to contract scale.
100% of Defra’s digital and technology projects are assessed for sustainability impact, and provided guidance on best practice, mitigating actions and training materials.
Outcome for 2030
Strive to deliver digital and technology solutions that will provide environmental benefits and opportunities, which also outweigh any environmental costs. Understand and know how to reduce the impact that digital and technology services have on consumption of resources and water, and production of pollutants.

SO3: reduce natural resource use and improve our circular economy approach  
What we need to demonstrate
How Defra’s operations and services can contribute to a circular economy approach. This is most likely to be through services which provide, dispose of or utilise hardware, data centres and network infrastructure. We must explore opportunities to minimise our demand for primary resources and prevent unnecessary resource consumption and waste generation. This is through addressing consumption, re-use, re-distribution and re-cycling (buying better, using better and using longer).

Activities that will help us to deliver this objective by 2030
[footnote 15]

Limit our consumption of raw materials:

not creating new services or assets where existing ones meet the requirements
procure remanufactured devices by default where feasible (devices that have been refurbished through rebuilding or repair). Ensure quality and support from the vendor for refurbished devices
where this is not possible, procure devices made from recycled materials (devices made from existing material) where this is available on the market
procure devices which are designed for maximum circularity (for example, designed to be modular)
Increase the longevity of devices within Defra:

where feasible procure modular and repairable devices to support reuse and longevity
improve our capacity to refurbish, re-distribute and re-use hardware within Defra group to extend the life of usable devices
use analytics-driven performance-based device refreshes: ensuring we are extending device lifetimes to their full potential, rather than implementing standard lifecycle refreshes
Manage our e-waste:

follow the waste hierarchy for ITAD (IT asset disposal) by ensuring devices at the end of life in Defra are firstly reused, then components are recycled and landfill or incineration is used as a last resort
maintain our commitment to zero to landfill disposal for all electronic waste
work with GIO asset team and ITAD disposal partners to ensure all assets [footnote 16] are tracked through their lifecycle at Defra
longer term aim: plan to have onshore recycling or recovery of equipment so materials remain in the UK
Embed circularity in our supply chain:

procure from suppliers that align with our circular economy principles
collaborate with suppliers to develop circular solutions, harnessing their insights and innovations to better understand how resources and waste can be minimised in service delivery
Go beyond hardware in our approach:

extend our circular approach to digital solutions – re-using existing applications, code and platforms. For example, using the common delivery platform to re-use hosting and coding capacity and application of the Greener Service Principles
Key Performance Indicators
0% to landfill target achieved.
80% end user devices in Defra group by 2030 are refurbished and/or re-manufactured.
32% target for reduction in e-waste volumes over the 5 year period.
% annual increase in re-use, re-distribution and repair activities for existing hardware devices.
100% traceability of ICT hardware (devices and networks) at end of life through e-waste reporting requirements.
Outcome for 2030
To be an organisation that operates with circularity at the forefront, supporting the government’s priority to move Britain from a linear to a zero waste economy. This would be demonstrated by a reduction in the volume of resources consumed.

SO4: reduce social risk and deliver social value
What we need to demonstrate
We need to manage the areas of our operation supply chain that are high risk for social inequality, and exploitation. We need to ensure that our public spending generates additional economic, social and environmental benefits.

Activities that will help us to deliver this objective by 2030
Address social risk in procurement:

ensure suppliers comprehensively address modern slavery (including forced labour and child labour) in the supply chain. Where appropriate to do so, suppliers must complete the modern slavery assessment tool and will be encouraged to go beyond this for high-risk contracts, such as by working with the Slave-Free Alliance
ensure suppliers to uphold fair labour practices, including safe working conditions, reasonable working hours, and fair wages throughout their supply chains. Suppliers should provide evidence of labour rights compliance and continuous improvement initiatives
require suppliers to address the EDI (Equality, Diversity, and Inclusion) risks in technology and address through adopting EDI policies and conducting equality impact assessments
for large (over £5 million) contracts, encourage suppliers to address the technology skills shortage risk through investing in workforce development programs, such as apprenticeships, training, and upskilling initiatives. Suppliers should demonstrate active efforts to build a resilient and future-ready talent pipeline
ensure that public-facing systems procured are accessible, inclusive, and user-friendly for all users, including those with limited digital skills or access. Suppliers should demonstrate commitment to digital accessibility standards and support initiatives that promote equitable access to digital services
Strengthen social value in procurement:

embedding social value as a core component of procurement: expecting suppliers to make significant commitments beyond the core contract alongside their service delivery
Hold suppliers accountable:

using a risk assessment tool continuously monitor and challenge our strategic suppliers [footnote 17] on their social risk, including modern slavery and sustainable resourcing mapping
establish clear reporting expectations for suppliers to track their progress on social value commitments
establish a social value board for strategic suppliers where they discuss their commitments and progress
Build knowledge and capability:

provide learning and training resources for DDTS staff on social risks, impacts, and how to identify them in procurement and service delivery
update digital sustainability guidance to be included in DDTS business cases, ensuring social risk and value is considered from the outset
look at opportunities for Defra group to address social value / risk ourselves (as well as through suppliers)
Identify and address social risks:

long term aim:  produce a social risk analysis for DDTS that identifies key areas of risk associated with Defra’s digital and technology services
as part of this, conduct a gap analysis across our contracts and services to identify where we are missing key social risks in our overall provision
Key Performance Indicators
100% staff involved in sustainability and social value tender evaluations in DDTS technology procurement to complete training on sustainability.
100% contracts with DDTS strategic suppliers to include social value commitments.
Of the  Procurement Act 2023, 10% minimum weighting of social value (economic, social and environmental benefit, and risk management) for contracts over £5 million, 5% must be on social benefit and risk mitigation.
Outcome for 2030
To be a sustainability focused organisation that understands and mitigates sustainability related risks, whilst striving to secure additional benefits through public spending activities (social value). 

SO5: increase supply chain transparency and accountability
What we need to demonstrate
Why Defra’s digital and technology supply chain transparency is critical to positive environmental, social and economic sustainability outcomes. That we are committed to holding our suppliers to account on this. We need to better understand the risks and global challenges within our supply chain, and what measures can be taken to reduce them.

Activities that will help us to deliver this objective by 2030
Early sustainability engagement:

engage with projects and procurements early on to allow for sustainable design of solutions
Transparent and responsible procurement:

where proportionate and relevant to do so, adopt early market engagement activities across procurements to gage market capability to deliver on sustainability ambition
set robust vendor prequalification criteria for contractual sustainability performance specific to the procurement
consistently use proportionate and relevant sustainability weighting and evaluation processes in our contracts and procurement activity.  For Tier 1 [footnote 18] contracts, Defra will go beyond Procurement Act requirements to apply a 15% weighting on sustainability
help the business and project teams better understand where sustainability impacts are present in the supply chain so that they can be better mitigated
work with commercial to formalise our roles in procurement, with the sustainability team acting as a subject matter expert and the commercial team providing procurement expertise
work with commercial and group infrastructure operations to create and maintain a resource bank with category guides and best practice sustainability questions, requirements and reporting measures
work with commercial to produce self-service guidance for project and commercial teams
ensure direct award procurements include sustainability as a non-functional requirement, where proportionate and relevant to do so
work with commercial to ensure that employees involved in procurement have the right skills and knowledge to set sustainability and social value requirements and evaluate procurement questions
Consistent supplier and service management:

embed the supplier engagement plan for DDTS strategic suppliers [footnote 19], ensuring consistent and structured engagement with suppliers on sustainability performance
better support and upskill service owners and suppliers to manage sustainability and social value on their accounts / contracts
establish a regular reporting cycle for Defra group to hold suppliers and services and DDTS accountable
create best practice guides for how to manage sustainability through the lifecycle of a contract
Supply chain transparency

work with suppliers and the GDSA members to better understand our scope 3 emissions data and supply chain risks to improve our reporting capability
be transparent and open about our own supply chain and honest about where we have gaps in our understanding
Key Performance Indicators
All strategic [footnote 20] suppliers provide reporting information, as per the relevant contract to Defra.
Quarterly engagement with 100% of strategic suppliers.
Quarterly RAG rating provided for 100% of strategic supplier.
100% of Tier 1 procurements include at minimum 15% social value (economic, environment and social) weighting, above the 10% required by the Procurement Act, unless Defra determines it would not be proportionate or relevant to do so.
Outcome for 2030:
To be an organisation that has a clear understanding of its supply chain and what sustainability risks are present. To have plans in place to mitigate these risks and leverage opportunities.

SO6: improve resilience to climate and environment risks
What we need to demonstrate
How Defra’s digital and technology operations and services may be at risk from an imminently changing climate, environmental risks and critical minerals shortages. Be able to demonstrate what adaptation measures we are taking to mitigate the threats posed by climate and environmental change.

Activities that help to deliver this objective by 2030
Improve our understanding of the risks:

through systems mapping better understanding the risks that climate change and environmental risks and critical minerals shortages present to DDTS operations
work to understand the risk proximity, impact and likelihood for Defra group technology across main climate risk factors
working with suppliers to understand the risk to existing cornerstone digital infrastructure
Mitigating the risks:

ensuring business continuity teams are aware of the risks and have climate risk factors embedded in business continuity plans
ensure business continuity teams are aware of the economic risks of future critical minerals shortages and these are embedded in business continuity plans
ensure our systems are stress tested for climate outcomes
require procurement of climate-resilient IT infrastructure, for example that relies on less water and operates under high temperatures
collaborate with the supply chain to develop documentation and training on resilience
collaborating with Defra SCoE (sustainability centre of excellence) on group level climate change risk assessment, with a focus on digital technology and IT services
educating the DDTS workforce on the potential risks that climate change and environmental risks pose to Defra’s digital and technology services, to build capability and resilience
Key Performance Indicators:
100% of strategic suppliers required to demonstrate awareness of climate risk to the services they provide and have appropriate mitigation and adaptation plans in place (through a risk register assured by Defra digital sustainability team).
Complete analysis by 2030 to identify which DDTS operations are most at risk from climate change and environmental risks and produce mitigation/adaptation plans for keystone digital infrastructure.
Climate change and environmental risks are built into and considered as part of 100% of project risk registers.
Outcome for 2030
To understand the threats and risks that climate change and other environmental risks pose to our operations and services. Improve our resilience and preparedness to these risks through targeted adaptation measures and business continuity plans.

Embed digital sustainability as business as usual (BAU)
Embedding digital sustainability into our business as usual operations may mean decision making includes competing priorities. For example, reducing water usage on a service might increase carbon emissions. Or a more sustainable solution might involve higher upfront costs. It’s essential that we have clear processes and training in place to ensure sustainability considerations can be consistently factored in informed decisions to support long term sustainability goals.

What we need to demonstrate
That sustainability can and should be a core priority for everyone – no matter what their role is or what type of work they do. Sustainability improvements will be realised through the active contributions of the whole workforce: those who lead and drive change, those who deliver and manage digital capabilities, and end-users who are aware of their own sustainability responsibilities.

Activities that will help us to deliver this objective by 2030
Training, upskilling and knowledge sharing:

deliver digital sustainability training for any Defra group staff working on a corporately provided digital service or technology, tailored to their role and refreshed at least annually
include training as part of DDTS induction
upskill developers, user designers and architects to design and build systems and services with efficiency, limiting unproductive service usage and consumption
developers, dev ops, testers, architects, site reliability engineers (SRE’s), product managers and user designers to complete the Green Software Foundation course
where relevant and proportionate ensure suppliers and contractors have the relevant green IT skills for their role, by including this as a requirement in procurement or through ongoing supplier engagement.
embed digital sustainability across Defra group more widely, with an emphasis on individual responsibility and through leveraging existing programmes and tools such as the Digital Confidence for All programme
work with our arm’s length bodies to embed digital sustainability in additional digital confidence courses
launch and maintain a digital sustainability awareness SharePoint site to be a single source of truth and provide latest information for upskilling
prepare and deliver campaigns, leveraging current tooling, to ensure end-users across Defra group are empowered to make sustainable decisions
work with the Chief Data Officer to enable staff to manage their data efficiently and sustainably
work with the Information Management team to ensure that information is managed effectively throughout its lifecycle and the accumulation of low value information is avoided
produce a procurement guide for DDTS colleagues that supports sustainable procurement activities (for example, defining requirements and drafting contract clauses, completing evaluations)
produce a contract manager guide for how to manage sustainability through the lifecycle of a contract
Make sustainable design the default for service delivery:

work with the service design profession to embed greener service principles across service delivery in Defra and incorporate these into the GDS Service Standard and Service Manual
develop a methodology for projects and services to use to show sustainability costs and benefits as part of their business case and forecasting
work with the Defra investment committee to ensure digital sustainability is embedded into all business cases
continue developing and maintaining a set of non-functional requirements relating to digital sustainability for Defra group
work with service owners to develop sustainability statements for Defra services within the priority services set out by GDS
ensure that all Defra group AI use cases comply with the Defra group AI position statement and checklist
refine the digital sustainability project assessment process, giving us a view of impact across the portfolio, with tiered engagement from the digital sustainability team for highest impact projects
ensure digital sustainability is embedded into the enterprise architecture guardrails
ensure that digital sustainability is considered as part of PAB and other project approvals
improve how we calculate sustainability cost and benefits to more effectively demonstrate net benefit where appropriate
create a 15th service standard for Defra projects that is environmental sustainability focused.  Services to publish sustainability statements as part of their dashboards
Engaging across Defra group:

senior leaders across DDTS to understand how their functions can deliver digital sustainability
establish a network of digital sustainability advocates in each DDTS function to support implementation of this strategy within their function
develop a digital sustainability Community of Practice focused on digital delivery.
embed digital sustainability within Defra group’s performance reporting and risk management processes
engage with the innovation team to horizon scan and identify future technologies that  may present risks or opportunities
engage with the ALB (arms length bodies) transformation boards to ensure digital sustainability is considered in decision making
embed digital sustainability into our work to deliver the digital and data transformation strategy and new DDTS operating model – including establishing roles and responsibilities for considering digital sustainability risks and opportunities within Delivery Groups
Balancing sustainability and other business requirements:

identify areas and trade-offs where sustainability and business operations come into conflict with each other (such as mirroring applications to improve resilience, or pre-loading data to improve user experience)
identify the decision pipeline for when trade-offs are made and ensure sustainability is factored into the decision
Key Performance Indicators
75% of DDTS staff completing digital sustainability training annually once mandatory training in place).
100% of projects to have completed sustainability assessments and demonstrate that they have taken appropriate actions from the greener service principles.
Quarterly update provided to DDTS Exec Board on sustainability with progress update from each function.
GDS priority services to have sustainability statements in place by 31 March 2027.
100% of technology procurements to include relevant and proportionate sustainability criteria under social value weighting as part of supplier selection.
An annual improvement in digital sustainability knowledge as reported in the annual IT survey.
Outcome for 2030
Be an organisation where sustainability is a core priority and an integral part of operations, services and activities – not treated like an ‘added extra’. Lead by example through having a workforce that are aware of digital sustainability and how their work can contribute to it.

Functional leadership responsibilities within DDTS
Chief Digital Information Officer – Senior Responsible Owner of Digital Sustainability
Responsible for:

Provision and development of digital and technology services used in Defra group. Providing leadership to functional managers across DDTS.

Leadership role:

To act as a role model for digital sustainability across government. To ensure leaders within DDTS feel informed and empowered to embed sustainability within their functional areas of responsibility.

Accountable for:

Prioritising digital sustainability amongst SCS counterparts across government.
Holding DDTS Executive Board accountable for commitments on digital sustainability within their respective functions.
Director Digital Delivery
Responsible for:

Overseeing portfolio management, portfolio delivery, user-centred design, development and testing and technical services. 

Leadership role:

To help embed sustainability into portfolio delivery, ensuring all projects are assessed for sustainability impacts and the design and development of technical services includes sustainability.

Accountable for:

Approving and championing the new digital sustainability assessment process including tiered self-serve / direct support.
Identifying potential projects as early adopters of Greener Service Principles, to serve as exemplars of Defra leading the way across Government. Encouraging their use across Defra in design, delivery and operation of services.
Supporting inclusion of sustainability within delivery supplier selection process.
Being accountable for KPI of ensuring all projects complete digital sustainability assessment.
Ensuring Common Delivery Platform is an enabler for improved sustainability outcomes.
Support for a 15th sustainable service standard for Defra.
Director Group Infrastructure and Operations
Responsible for:

Overseeing service operations, portfolio delivery, supplier and service management, platforms and service design.

Leadership role:

To ensure sustainability is prioritised in supplier and service management, service design and operations. To improve sustainable procurement performance and standards.

Responsible for:

Ensuring that projects in the GIO delivery portfolio to engage with the digital sustainability team effectively and proportionally.
Ensure that all procurements in the GIO delivery portfolio include minimum sustainability as a 15% weighting, unless explicitly agreed with the digital sustainability team.
Ensuring that all DDTS suppliers deliver their contracted sustainability and social value performance and provide additional value to Defra’s work on the topic, including by working to the agreed supplier engagement plan.
Ensuring that GIO owned infrastructure and services are as sustainable as possible, reducing emissions and e-waste and future metrics as developed year on year.
Ensure that CCoE and off-premise data centre teams embed GreenOps principles in their delivery and maintenance of services.
Chief Digital Officer
Responsible for:

Overseeing the digital strategy, engagement and transformation teams.

Leadership role:

To ensure ‘sustainability as business as usual’ is well embedded into DDTS strategies and transformation activities, without being treated as an added extra.

Responsible for:

Embedding sustainability into delivering our digital and data transformation strategy.
Embedding digital sustainability across the new operating model, including in PAB (portfolio assurance board) and other approvals process.
Chief Technology Officer
Responsible for:

Overseeing technical innovation, strategic architecture, data information and delivery, and business management.

Leadership role:

To ensure that sustainability is considered as part of innovation activities and the design of architecture related services.

Responsible for:

Being a figurehead for sustainable innovative technology, for example, sustainable use of artificial intelligence.
Horizon scanning for emerging sustainability risks and best practice in technology.
Ensuring digital sustainability and GSPs are reflected across enterprise architecture framework and guardrails.
Embedding sustainability in data governance standards.
Head of Defra Senior Security Adviser (SSA)
Responsible for:

Defra group Security, Group Corporate Services Crisis Management, Business Continuity, Security risk, assurance, and compliance.  

Leadership role:

To collaborate with sustainability team to find common ground between Defra group Security capabilities and sustainability.  To work with colleagues across DDTS to support the sustainability agenda and ensure outcomes are secure, sufficient and sustainable.

Responsible for:

Supporting our 0% to landfill ambition.
Collaborating with sustainability to maximise harmony between sustainability and security.
Supporting sustainable by design alongside secure by design.
Ensuring resilience to climate risk is embedded into business resilience plans.
Chief Operating Officer
Responsible for:

Overseeing governance and risk, business planning, resourcing and reporting, assurance, and digital sustainability.

Leadership role:

To support employees across DDTS to access sustainability learning and training resources. To ensure climate risks are captured in reporting and assurance processes, and business continuity plans in place.

Accountable for:

Providing communications support for digital sustainability across DDTS.
Ensuring sustainability and climate risks is captured in governance and risk activity and business continuity plans.
Helping facilitate learning and development training resources on sustainability for all DDTS staff.
Providing ongoing support to the digital sustainability team.
Delivery Group Leads
Responsible for:

Overseeing the delivery groups to deliver against common outcomes for Defra group policy.

Leadership role:

To ensure that sustainability is considered alongside policy and delivery of the ALB services.

Responsible for:

Advocating sustainability across delivery.
Encourage use of GSPs in design, delivery and operation of services.
Ensure all live services are continuously evaluated to comply with the GSPs as far as possible.
Next steps
Timescales
This Digital Sustainability Strategy covers the 5-year period from 2025 to 2030. This aligns with the Greening Government Commitments refresh from 2025 to 2030, including the Greening Government: ICT and digital services strategy which will be updated and published after a review to ensure they remained aligned with government priorities.

The Defra group Sustainability Strategy goes beyond this target year, to 2033. We will ensure that we continue to align with the Sustainability Centre of Expertise (who own the strategy within Defra) so that our timescales, activities and progress is aligned.  Beyond 2030 we’ll assess our progress, and set new and ambitious targets reflecting the policy, science, and technological development at the time, aligning with net zero.

Governance and accountability
Overall responsibility for this Digital Sustainability Strategy lies with the Executive Committee (ExCo), but Defra’s CDIO and SRO for Digital Sustainability is the sponsor. As SRO they will ensure that the digital sustainability strategy is implemented, and that associated actions plans link to our vision, ambitions and strategic objectives.

This strategy will help us further embed digital sustainability within Defra. To make sure the deliverables and KPIs from this strategy are achieved, quarterly progress updates will be given to Defra’s CDIO and our Executive Board. Risks to meeting our KPIs will be escalated to Group Corporate Services Board as needed. In addition, an annual progress summary will be shared with Defra employees, to demonstrate progress against the commitments set out here.

Ownership / queries
This strategy is owned by the Defra Digital Sustainability Team. Any questions about the contents should be directed to sustainableict@defra.gov.uk

October 2025